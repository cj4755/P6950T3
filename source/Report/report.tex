\documentclass{article}
\usepackage[utf8]{inputenc}
\usepackage[english]{babel}
\usepackage{amsmath}
\usepackage{natbib}
\usepackage{graphicx}

\begin{document}

{\centering

\rule{\textwidth}{1.6pt}\vspace*{-\baselineskip}\vspace*{2pt} 
\rule{\textwidth}{0.4pt}\\[\baselineskip] 
{\LARGE Growing Degree-day}
\rule{\textwidth}{0.4pt}\vspace*{-\baselineskip}\vspace{3.2pt}
\rule{\textwidth}{1.6pt}\\[\baselineskip] 

\vspace{20mm} %5mm vertical space
\scshape % Small caps
CMSC 6950 - Computer Based Research Tools and Applications \\ [\baselineskip]
Extended Assignment \\[\baselineskip] 
13th June, 2016 \\[\baselineskip] 
\vspace{20mm} %5mm vertical space
Submitted by \\[\baselineskip]
{\Large Mohammad Hassan \\ Ernest \\ Mehrzad \\ Yin Zhang \\ Rufai raji \\ Lutfor Rahman \\ Mohammad\par}
\vfill
{\itshape Memorial University of Newfoundland \\ St. John's, Canada.\par} 
}

\newpage

{\centering
  \section*{Abstract}
}

{\itshape In this report, we calculated the growing degree days (GDD) based on different Canadian city's weather history. Three major cities were considered, namely: St. John’s, Calgary and Montreal. The data collected was for the year 2015. The calculated GDD varies among the three cities considered. The city of Montreal had the largest GDD of 1448 $^{\circ}$C for the year 2015 while St. John's had the least GDD of 501 $^{\circ}$C. 
}

\newpage
\tableofcontents
\newpage

\section{ \bf Introduction}
Growing degree days (GDD) is simply a predictive tool used in horticulture for accessing crop maturity and development. This predictive tool is helpful in providing adequate knowledge for timing the important growth stages of plants, as well as in assessing the suitability of a region for growing specific crops of interest. As such, GDD units are a frequently used weather-based indicator for plant development assessment. The GDD calculation within a period of time is dependent on heat accumulation and is typically measured from the start of the growth cycle of a plant. It is usually defined relative to a base reference temperature which varies among crops based on their life cycles. The total GDDs over a growing season is related to plant development. The development of plants depends on the accumulation of heat. Since cool season plants have a lower reference temperature, they accumulate GDDs faster than warm season plants. Unless plants are overly stressed by drought or pests, the total GDDs can be used to predict when a crop will reach maturity. Corn, for example, requires 1360 GDD units to mature. GDDs can be computed using climatic information for any location. That computation, along with data on soil, water, and minimum and maximum temperatures, helps suggest which crops will grow best in a given region.\\ [\baselineskip] GDD has units of temperature (i.e., $^{\circ}$C or $^{\circ}$F) and is calculated by subtracting the base temperature from the average of the daily temperatures (i.e. maximum and minimum temperatures). The equation for calculating the GDD, Eqn.(\ref{eqn:gdd}), is given as follows:

\begin{equation}
\textrm{GDD} = \left(\frac{T_{max} + T_{min}}{2}\right) - T_{base}
\label{eqn:gdd}
\end{equation}

\noindent where {$T_{max}$} is the daily maximum temperature and {$T_{min}$} is the daily minimum temperature. {$T_{base}$} is the base temperature, which is the minimum temperature required for the growth of a particular crop. For the GDD calculations reported in the study, {$T_{base}$} is considered to be 10 $^{\circ}$C.
In this report, we present a summary of the results obtained from calculating the annual GDD in 2015 for three cities in Canada; namely: St. John's, Montreal and Calgary. As expected, the calculated GDDs varies for the selected cities. 


\section{ \bf Methodology}
\subsection{Data Collection}
The data for the daily historical temperatures for the selected cities were downloaded from: http://climate.weather.gc.ca. From the collected data, the minimum and maximum daily temperature values all year round were extracted for the selected cities. Daily GDD values were then computed and the results were analyzed and plotted as graphs. A summary of the approach adopted in data collection and analysis is presented below:

i. Download daily historical data for the three selected cities 

ii. Calculate Daily GDD for the cities for the time range selected 

iii. Compute Cumulative GDD for the cities within the selected time range

iv. Create a plot showing variations in annual GDD cycle for the chosen cities within the selected time range 

v. Create a plot demonstrating accumulated GDD vs time for the chosen cities within the time frame selected 


\subsection{Growing Degree Day Calculation}
Growing Degree Day (GDD) are calculated by taking the average of the daily maximum and minimum temperatures compared to a base temperature, $T_{base}$, (usually 10$^{\circ}$C). As an equation: \vspace{5mm}

{\centering
$ GDD = (T_{max} - T_{min} /2) - T_{base} $\\ [\baselineskip]
}

Here, if the mean daily temperature is lower than the base temperature then $GDD = 0$.\vspace{5mm}

GDDs are typically measured from the winter low. Any temperature below $T_{base}$ is set to $T_{base}$ before calculating the average. Likewise, the maximum temperature is usually capped at 30$^{\circ}$C because most plants and insects do not grow any faster above that temperature. However, some warm temperate and tropical plants do have significant requirements for days above 30$^{\circ}$C to mature fruit or seeds.\vspace{5mm}

For example, a day with a high of 23$^{\circ}$C and a low of 12$^{\circ}$C (and a base of 10$^{\circ}$C) would contribute 7.5 GDDs.\vspace{5mm}

\[ \frac {23+12}{2}-10=7.5 \] \par

A day with a high of 13$^{\circ}$C and a low of 10$^{\circ}$C (and a base of 10$^{\circ}$C) would contribute 1.5 GDDs.\vspace{5mm}

\[ \frac {13+10}{2}-10=1.5 \] \par

\noindent
\section{ \bf Software Architecture}
\subsection{Tools Used}
\subsubsection{Programming Language}
\subsubsection{Version Control}
\subsection{Components}
\subsubsection{Makefile}
\subsubsection{Plots}
\subsubsection{Report}
\subsubsection{Presentation}

\section{ \bf Core Tasks}
\subsection{Download daily historical temperature data for several cities}
\subsection{Create a plot showing an annual cycle of min/max daily temperatures. Do this for at least three selected Canadian cities.}
\subsection{A command line program that takes arguments.}
This program should calculate the GDD. Internally your program should handle the command line arguments and implement the actual calculation as one or more functions. The output from this program needs to be persistently stored. Your choice on how to implement this storage. Later steps in your work flow must use the results of these calculations.
\subsection{Create plots showing accumulated GDD vs time for selected cities}

\subsection{Use version control (git) and collaboration tools (GitHub)}
\subsection{Create a LaTeX report summarizing the results of your project}
\subsection{Create a web based presentation for your results}
\subsection{Implement your entire workflow as a Makefile. Ensure that your entire project is reproducible}
\subsection{Create a test-suite (using the Python package nose) to demonstrate your GDD calculation works as intended}
\subsection{Project should include adequate documentation both with your source code and Readme.md file}


\section{ \bf Optional Tasks}
\subsection{Create an plot showing GDD like the example below for selected Canadian cities}
\subsection{Create a map showing effective growing degrees over both all of Canada and only for the island of Newfoundland.}
\subsection{Explore how GDD calculation depends on the choice of $T_{base}$. show your results for either selected cities or create maps}
\subsection{Create standalone bokeh plots embeded in your HTML presentation}
\subsection{Create a bokeh server plot so that you can look at the accumulated GDD for any city in Canada.}



\section{Conclusion}
The daily GDD of three cities in Canada (St. John's, Montreal and Calgary) were calculated for the year 2015. The summation of the daily GDD units were used to compare heat accumulation in these cities. The results obtained indicate that the city of Montreal had the highest cumulative GDD of 1448 $^{\circ}$C for the year 2015 among the selected cities. The city of St. John's had the lowest GDD of 501 $^{\circ}$C, while Calgary had a moderate GDD of 796 $^{\circ}$C.

\bibliographystyle{plain}
\bibliography{./source/Report/references.bib}

\begin{figure}[h!]
\centering
\includegraphics[scale=.5]{./Plots/GDD_Plot.png}
\caption{GDD Cumulative Plot}
\label{fig: GDD Plot}
\end{figure}

\end{document}
